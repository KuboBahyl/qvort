\title{{\sc qvort}\\
{\normalsize A quantised vortex code}}
\author{\normalsize
        Andrew Baggaley\\
        \normalsize  School of Mathematics and Statistics\\
        \normalsize  Newcastle Univeristy\\
        {\normalsize  U.K.}\\
        \normalsize   a.w.baggaley@gmail.com
}
\date{\normalsize   \today}

\documentclass[12pt]{article}
\usepackage{natbib}
\usepackage{graphicx}
\usepackage{hyperref}
\newcommand{\bs}{\mathbf{s}}
\newcommand{\br}{\mathbf{r}}
\newcommand{\bu}{\mathbf{u}}
\newcommand{\appsection}[1]{\let\oldthesection\thesection
  \renewcommand{\thesection}{Appendix \oldthesection}
  \section{#1}\let\thesection\oldthesection}
\begin{document}
\maketitle
\pagestyle{plain}
\pagenumbering{roman}
\tableofcontents
\newpage
\clearpage
\pagenumbering{arabic}
\section{Introduction}\label{Sec:int}
  Welcome to the documentation for the {\sc qvort} quantised vortex code.
  The code is written in well commented Fortran, arranged in a modular structure.
  The primary aim of this documentation is to highlight which module(s) are involved in the various sequences of the code, 
  and also the parameters which can be set in the {\it run.in} file.
  To run {\sc qvort} you need to be on a {\it UNIX}-type system ({\it Make} and {\it bash} are heavily used) with a Fortran compiler.
  All the Fortran code can be found in the directory {\it src}.
  MATLAB is used for post-processing, although it is probably very straightforward to convert the MATLAB scripts (found in {\it bin}) to work with {\it Octave}, freely available under the GNU license.
  Please do not hesitate to contact with any queries, or if you are interested in helping to further develop the code.

\section{Equations}\label{Sec:eqn}
  The {\sc qvort} code is a vortex-filament code, a technique pioneerd by Schwarz in the early 1980s. 
  Each line-vortex in the system is discretised into a number of points which are evolved according to an equation of motion.
  Points along the line are added (or removed) if the vortex is streched (or compressed).
  If any two lines become very close (a distance less than the separation along the line) then the filaments reconnect, changing the topology of the system.  
  More precisely we define our vortex filament as a three diemensional curve $\bs=\bs(\xi,t)$.
  Here $\xi$ represents arc-lengths and $t$ is time.
  We can construct the tangent vector $\bs'$, then normal vector, $\bs''$,
  and the binormal vector, $\bs' \times \bs''$ by taking numerical derivatives.
  Note $\bs'=d \bs/d\xi$, and so on.
  We denote the velocity at a point $\bs$ as $\bu(\bs)$ and consider the velocity field to be made up of two parts,
  \begin{equation} 
     \bu=\bu_\mathrm{s}+\bu_\mathrm{n},
  \end{equation} 
  where $\bu_\mathrm{s}$ is the superfluid component of the velocity field (i.e. motion induced by vortices),
  and $\bu_\mathrm{n}$ is the normal fluid component.
  In the code there are various options for both the normal and superfluid components.
  For the superfluid component one can choose between the local induction approximation (LIA),
  \begin{equation}\label{Eq:LIA}
    \bu_\mathrm{s}=\beta \bs' \times \bs'', \qquad \beta=\frac{\Gamma}{4\pi} \ln \left( \frac{R}{a} \right),
  \end{equation} 
  where $\Gamma$ is the quantum of circulation (a parameter which can be set in {\it run.in}), 
  $R$ is the radius of curvature ($1/|\bs''|$), and $a$ is the vortex core size (fixed $10^{-8}$cm).
  We note that the time per iteration with this method scales like ${\cal O}(N)$, where $N$ is the total
  number of particles in the simulation.

  A second option is to solve the de-singulised Biot-Savart integral,
  \begin{equation}\label{Eq:BS}
    \bu_\mathrm{s}=\beta' \bs' \times \bs''+ \frac{\Gamma}{4\pi} \int_{\ell'} \frac{(\bs-\br)\times d\bs}{|\bs-\br|^3},
    \qquad \beta'=\frac{\Gamma}{4\pi} \ln \left( \frac{\sqrt{\ell_{i}\ell_{i-1}}}{a}\right).
  \end{equation}
  $\ell'$ represents the full system with the local region around the point of interest removed (to avoid division by 0).
  This local contribution enters back in as a binormal vector with a magnitude affected by the size of local region $\ell_i$/$\ell_{i-1}$.
  Whilst a more realistic scheme the downside of using the full integral comes in the scaling of the time per iteration with $N$, which is quadratic (${\cal O}(N^2)$).
      
  The current options for the normal fluid are zero, $\bs_\mathrm{n}=0.$, a constant $x$-flow, 
  $\bs_\mathrm{n}=(\mathrm{const},0,0),$ and the ABC flow,
  \begin{equation} 
    \bs_\mathrm{n}=
  \end{equation} 
  These two velocities combind together to give the total equation of motion
  \begin{equation}
    \frac{d \bs}{dt}=\bu_\mathrm{s}+\alpha \bs' \times (\bu_\mathrm{n}-\bu_\mathrm{s})
    +\alpha' \bs' \times \left[ \bs' \times (\bu_\mathrm{n}-\bu_\mathrm{s})\right]
  \end{equation}  
\section{Numerical schemes}\label{Sec:num}
  At a particular particle on a filament, with position $\bs_i$, define the distance to the particle infront ($\bs_{i+1}$)
  as $\ell_{i+1}$ and the distance to the particle behind ($\bs_{i-1}$) as $\ell_{i-1}$.
  By infront/behind we refer to the particles next/previous along the filament.
  We can construct finite difference approximations to the first and second derivatives by taking Taylor's series expansions.
  The resulting forms are as follows,
  \begin{equation}
    \frac{d \bs_i}{d \xi}=\frac{\ell_{i-1}\bs_{i+1}+(\ell_{i+1}-\ell_{i-1})\bs_i+\ell_{i+1}\bs_{i-1}}
    {2\ell_{i+1}\ell_{i-1}}+{\cal O}(\ell^2),
  \end{equation}
  \begin{equation}
    \frac{d^2 \bs_i}{d \xi}^2=\frac{2\bs_{i+1}}{\ell_{i+1}(\ell_{i+1}+\ell_{i-1})}-\frac{2\bs_i}{\ell_{i+1}\ell_{i-1}}+\frac{2\bs_{i-1}}{\ell_{i-1}(\ell_{i+1}+\ell_{i-1})}+{\cal O}(\ell^2).
  \end{equation}
  We must also use some numerical scheme to integrate in time
\end{document}
